%&Prebuild
\documentclass[titlepage]{report}
\usepackage[BibLaTeX,noParIndent,InlineTodonotes]{multipack}



\title{Computational Materials Science - Molecular Dynamics}
\author{Schippers, C.F.}

\newcommand\inputpgf[2]{{
		\let\pgfimageWithoutPath\pgfimage 
		\renewcommand{\pgfimage}[2][]{\pgfimageWithoutPath[##1]{#1/##2}}
		\input{#1/#2}
	}}
	
	\newcommand\pgffigure[1]{{
			\beginpgfgraphicnamed{PGFFigures/#1}
			\inputpgf{Figures}{#1.pgf}
			\endpgfgraphicnamed
		}}

\definecolor{codegreen}{rgb}{0,0.6,0}
\definecolor{codegray}{rgb}{0.5,0.5,0.5}
\definecolor{codepurple}{rgb}{0.58,0,0.82}
\definecolor{backcolour}{rgb}{0.95,0.95,0.92}

\lstdefinestyle{mystyle}{
	backgroundcolor=\color{backcolour},   
	commentstyle=\color{codegreen},
	keywordstyle=\color{magenta},
	numberstyle=\tiny\color{codegray},
	stringstyle=\color{codepurple},
	basicstyle=\footnotesize\ttfamily,
	breakatwhitespace=false,         
	breaklines=true,                 
	captionpos=t,                    
	keepspaces=true,                 
	numbers=left,                    
	numbersep=5pt,                  
	showspaces=false,                
	showstringspaces=false,
	showtabs=false,                  
	tabsize=2
}

\lstset{style=mystyle}

\csname endofdump\endcsname

%\syntaxonly
\date{\today}
\pgfrealjobname{Report}

\begin{document}
\setAbstract{}
\inserttitletoc

\listoftodos
\newpage


\chapter{Theoretical exercises}
\section{Exercise 1}
A Gaussian polymer coil with $ N $ monomers with a distance $ a $ between each other has a mean square end-to-end distance $ \left\langle R^2 \right\rangle = a^2 N $ and an approximate volume $ V \sim \left\langle R^2 \right\rangle^{\frac{3}{2}} = a^3 N^{\frac{3}{2}} $ \parencite{Rubinstein2003}. The actual volume which is occupied by the monomers with volume $ v $ of the polymer is given by $ V\sub{pol} = v N $. The volume fraction $ \phi = \frac{V\sub{pol}}{V} $ is then given by
\begin{equation}
	\phi \sim \frac{v N}{a^3 N^{\frac{3}{2}}} = \frac{v}{a^3} N^{-\frac{1}{2}}.
\end{equation}

As two-particle interactions occur when one particles is found close to another, which has probability $ phi $, one can assume that $ \nu_2 $ scales as
\begin{equation}
	\nu_2 \sim N \phi \sim N^\frac{1}{2}.
\end{equation}

Analogous, three-particle interactions occur only when two particles are found close to another particle, which has probability $ \phi^2 $, one can assume that $ \nu_3 $ scales as
\begin{equation}
	\nu_3 \sim N \phi^2 \sim N^0.
\end{equation}

As one can see $ \nu_2 $ increases with increasing $ N $ so becomes a significant contribution for a polymer, whereas $ \nu_3 $ remains small as it does not depend on $ N $.

\section{Exercise 2}
\subsection{Exercise 2a}\label{subsec:THEX2a}
The Lennard-Jones potential is given by
\begin{equation}\label{eq:LennardJones}
	U = 4 \varepsilon \left[ \left(\frac{\sigma}{r}\right)^{12} - \left(\frac{\sigma}{r}\right)^6\right].
\end{equation}

To find the minimum of this potential, take the derivative to $ r $ and put to zero for $ r = r\sub{min} $
\begin{equation}
	\left. \pd{U}{r} \right|_{r=r\sub{min}} = 4 \varepsilon \left[ -12 \frac{\sigma^{12}}{r\sub{min}^{13}} + 6\frac{\sigma^{6}}{r\sub{min}^{7}} \right] = 0.
\end{equation}

This solves to
\begin{equation}
	r\sub{min} = \sqrt[6]{2} \sigma.
\end{equation}

Putting this in $ U $ gives
\begin{equation}
	U\left(r = r\sub{min}\right) = - \varepsilon.
\end{equation}

\subsection{Exercise 2b}
$ D_e $ is the depth of the potential well. 
A Taylor expansion of the potential around $ l - l_0 = 0 $ gives
\begin{subequations}
	\begin{align}
		v(l) =& D_e \left[ 1 - \exp\left(-a (l-l_0)\right) \right]^2 \\
		\approx& a^2 D_e (l-l_0)^2 + \mathcal{O}\left((l-l_0)^3\right).
	\end{align}
\end{subequations}
So at small deviations from $ l_0 $ the Morse potential is approximately equal to a harmonic potential $ v(l) = k/2 (l-l_0)^2 $ with $ k = 2 a^2 D_e $.
At distances away from the equilibrium the Morse potential deviates from the harmonic potential as the Morse potential approaches the potential depth $ D_e $ asymptotically. 
The Morse potential and the Taylor expansion around $ l - l_0 = 0 $ are shown in \cref{fig:THEX2b}.
\begin{figure}[h!]
	\centering
	\pgffigure{THEX2b}
	\caption{Plot of the Morse potential and the Taylor expansion of the Morse potential around $ l-l_0 = 0 $.}
	\label{fig:THEX2b}
\end{figure}

\section{Exercise 3}
The characteristic frequency $ \omega $ of a harmonic spring with two masses is given by \cite[p. 164]{Taylor05}
\begin{equation}
	\omega = \sqrt{\frac{k\sub{spring}}{\mu}} 
\end{equation}
with $ k\sub{spring} $ the spring constant and $ \mu = \left(\frac{1}{m_1} + \frac{1}{m_2}\right)^{-1} $ the reduced mass. 
From this frequency, the wave-number $ k $ can be calculated with $ \omega = c k $, where $ c = 3\E{8}\unit{m \, s^{-1}} $ is the speed of light. This results in the values found in \cref{tab:THEX3wavenumbers}.
To convert the units to SI units, the conversion factor $ 1\unit{N/m} = 1.4393 \unit{kcal \, mol^{-1} \textrm{\AA}^{-2}} $ is used.
\begin{table}[h!]
	\centering
	\caption{Wave-numbers in $ \unit{cm^{-1}} $ for various bonds.}
	\label{tab:THEX3wavenumbers}
	\begin{tabular}{lr}
		Bond & Wave-number ($\unit{cm^{-1}}$) \\ 
		C-C & 4955.33 \\ 
		C=C & 7310.84 \\ 
		C=0 & 7257.00
	\end{tabular} 
\end{table}

\section{Exercise 4}
The total amount of boxes that has to be taken into account in $ N $-dimensions scales as $ C^N $.
For $ 1D $ systems this is known to be $ 3 $, for $ 2D $ systems this is $ 9 $ and for $ 3D $ systems this is $ 27 $, which leads to $ C=3 $.
So in a $ N $ dimensions the amount of periodic images is given by $ 3^N - 1 $.

\section{Exercise 5}
Using $ \vec{F} = m \ddot{\vec{r}} $, the virial $ \langle G \rangle $ can be calculated as
\begin{subequations}
	\begin{align}
		\langle G \rangle =& \left\langle \sum_i \vec{r}_i \cdot \vec{F}_i \right\rangle\\
		=& m \left\langle \sum_i \vec{r}_i \cdot \ddot{\vec{r}}_i \right\rangle.
	\end{align}
\end{subequations}

The average can be calculated using equation 2.1 from the lecture notes:
\begin{subequations}
	\begin{align}
	\langle G \rangle =& \lim\limits_{t\sub{obs} \rightarrow \infty} \frac{1}{t\sub{obs}} \int_{0}^{t\sub{obs}} m \sum_i \vec{r}_i \cdot \ddot{\vec{r}}_i \d t\\
	=& \lim\limits_{t\sub{obs} \rightarrow \infty} \frac{m}{t\sub{obs}} \left( \left. \left[\sum_i \vec{r}_i \cdot \dot{\vec{r}}_i \right] \right|_0^{t\sub{obs}} - \int_{0}^{t\sub{obs}} \sum_i \dot{\vec{r}}_i \cdot \dot{\vec{r}}_i \d t \right).
	\end{align}
\end{subequations}

As the first term vanishes as it is divided by $ t\sub{obs} $, which goes to $ \infty $. By using equation 2.1 from the lecture notes and $ \sum_i \left\langle \dot{\vec{r}}_i^{\;2} \right\rangle = N \left\langle \dot{\vec{r}}^{\;2} \right\rangle $, the remaining part can be rewritten:
\begin{subequations}
	\begin{align}
	\langle G \rangle =& - \lim\limits_{t\sub{obs} \rightarrow \infty} \frac{m}{t\sub{obs}} \int_{0}^{t\sub{obs}} \sum_i \dot{\vec{r}}_i^{\;2} \d t \\
	=& -m \left\langle \sum_i \dot{\vec{r}}_i^{\;2} \right\rangle\\
	=& -m N \left\langle \dot{\vec{r}}^{\;2} \right\rangle.
	\end{align}
\end{subequations}

As according to the equipartition theorem $ 3 k\sub{B} T = m \left\langle \vec{v}^{\;2} \right\rangle = m \left\langle \dot{\vec{r}}^{\;2} \right\rangle $, the virial is given by $ \langle G \rangle = -3 n k\sub{B} T $.\\

Moreover, the sum $ \sum_i \vec{r} \cdot \vec{F}_i\suprm{ext} $ can be calculated using $ \d \vec{F}_i\suprm{ext} = - P \d \vec{A})_i $:
\begin{subequations}
	\begin{align}
		\sum_i \vec{r} \cdot \vec{F}_i\suprm{ext} =& \sum_i \int \vec{r} \cdot \vec{F}_i\suprm{ext}\\
		=& \sum_i \int_{A_i} P \vec{r} \cdot \d\vec{A}_i.
	\end{align}
\end{subequations}

Using the Gauss theorem and the fact that $ \nabla \vec{r} = 3 $ this can be further calculated as
\begin{subequations}
	\begin{align}
		\sum_i \int_{A_i} P \vec{r} \cdot \d\vec{A}_i =& -P \sum_i \int_{V_i} \nabla \cdot \vec{r}_i \d V_i\\
		=& -3 P \sum_i V_i\\
		=& -3 P V.
	\end{align}
\end{subequations}

This results can be combined with $ \langle G \rangle = -3 n k\sub{B} T $ using $ \vec{F} = \vec{F}\suprm{int} + \vec{F}\suprm{ext} $:

\begin{subequations}
	\begin{align}
		-3 V P = \sum_i \vec{r} \cdot \vec{F}_i\suprm{ext} =& \left\langle \sum_i \vec{r}_i \cdot \vec{F}_i \right\rangle - \left\langle \sum_i \vec{r}_i \cdot \vec{F}_i\suprm{int} \right\rangle\\
		=& \langle G \rangle - \left\langle \sum_i \vec{r}_i \cdot \vec{F}_i\suprm{int} \right\rangle\\
		=& -3 N K\sub{B} T - \left\langle \sum_i \vec{r}_i \cdot \vec{F}_i\suprm{int} \right\rangle.
	\end{align}
\end{subequations}

This yields
\begin{equation}
P = \frac{1}{V} \left( N k\sub{B} T + \frac{1}{3} \left\langle \sum_i \vec{r}_i \cdot \vec{F}_i\suprm{int} \right\rangle \right).
\end{equation}

\section{Exercise 6}
\subsection{Exercise 6a}
The magnitude of the barrier separating trans and gauche isomeric states for polyethylene is $ 14 \unit{kJ / mol} $, which is equal to $ 5.65 k\sub{B} T $.

\subsection{Exercise 6b}
The Lennard-Jones time is defined as $ \tau\sub{LJ} = a\sqrt{\frac{m}{\varepsilon}} $.
For the specified model, $ a = 1 \textrm{\AA} $ is the size of the particle, $ m = 12 a.u. = 1.992 \E{-28} \unit{kg} $ is the mass and $ \varepsilon = 0.5 \unit{kJ / moll} = 8.3 \E{-20} \unit{J} $ is the interaction energy, so the Lennard-Jones time is given by $ \tau\sub{LJ} \approx 10^{-10} \sqrt{\frac{1.992 \E{-28}}{8.3 \E{-20}}} \approx 0.5 \unit{ps} $.
The rotational diffusion time is defined as $ \tau\sub{diff} = \frac{8 \pi \eta (l/2)^3}{k\sub{B} T} $.
With viscosity $ \eta = 1 \unit{mPa \, s} $ and temperature $ T = 300 \unit{K} $ and length $ l = 1 \AA $, this gives $ \tau\sub{diff} = \frac{8 \pi \E{-3} \E{-30}}{4.11\E{-21}} \approx 6\unit{ps} $.
An appropriate time-step would be $ \Delta t = \tau\sub{LJ} / 100 = 5 \E{-3} \unit{ps} $, as the Lennard-Jones time is the shortest time-scale. 

\section{Exercise 7}
The kinetic energy $ E\sub{k} $ and it's time derivative are given by
\begin{subequations}
	\begin{align}
		E\sub{k} =& \sum_{1=i}^{3N}\frac{1}{2} m_i v_i^2\\
		\dd{E\sub{k}}{t} =& \sum_{1=i}^{3N} m_i \dot{v}_i v_i.
	\end{align}
\end{subequations}

Using $	m_i \dot{v}_i = F_i + m_i \gamma \left(\frac{T_0}{T} -1 \right) v_i $, this can be written as
\begin{subequations}
	\begin{align}
		\dd{E\sub{k}}{t} =& \sum_{1=i}^{3N} \left( v_i F_i + m_i \gamma \left(\frac{T_0}{T} -1 \right) v_i^2 \right)\\
		=& \sum_{1=i}^{3N} v_i F_i + \gamma \left(\frac{T_0}{T} -1 \right) 2 E\sub{k}.
	\end{align}
\end{subequations}

Now using $ E\sub{k} = \frac{3}{2}N k\sub{B} T $ one can write
\begin{equation}
	\dd{E\sub{k}}{t} = \sum_{1=i}^{3N} v_i F_i + 2\gamma \left(\frac{3 N}{2} k\sub{B} T_0 - E\sub{k} \right).
\end{equation}


\chapter{Molecular Dynamics of a Simple Liquid}

\section{Exercise 1}
The model parameters are set in de main function.
The masses of the particles are determined by the variable \texttt{special}.
If $ \texttt{special} = 1 $ the particles have mass 1, but if $ \texttt{special} \neq 1 $ the particles have random masses following a Gaussian distribution centred around 1.

\section{Exercise 2}
\subsection{Exercise 2a}
The Lennard-Jones potential decreases sharply for values smaller than $ r\sub{min} = \sqrt[6]{2}\sigma \approx 1.122 \sigma $ and therefore induces a strong repulsive force, as can be seen in \cref{fig:MDSLEX2b}.
particles are initially spaced less then this $ r\sub{min} $, this force might cause particles to be displaced a multiple of the box size, which is unrealistic and thus leads to unrealistic behaviour.
Therefore, a spacing of $ r = 1.2 \sigma $ is a reasonable choice.

\subsection{Exercise 2b}
The Lennard-Jones potential is plotted in \cref{fig:MDSLEX2b}.
The minimum of the potential is $ U(r\sub{min}) = -\varepsilon $ with $ r\sub{min} = \sqrt[6]{2} \sigma $ as was calculated in theoretical exercise 2a (\cref{subsec:THEX2a}).
The asymptote for $ r \rightarrow 0 $ is $ U(r \rightarrow 0) \rightarrow \infty $ and the asymptote for $ r \rightarrow \infty $ is $ U(r \rightarrow \infty) \rightarrow 0 $.

\begin{figure}[h!]
	\centering
	\pgffigure{MDSLEX2b}
	\caption{Lennard-Jones potential.}
	\label{fig:MDSLEX2b}
\end{figure}

\section{Exercise 3}
\subsection{Exercise 3a}
The time-scale $ \tau $ can be defined as $ \tau = \sqrt{\frac{m \sigma^2}{\varepsilon}} $ with mass $ m $ in $ \unit{kg} $, distance $ \sigma $ in $ \unit{m} $ and energy $ \varepsilon $ in $ \unit{J} = \unit{kg \, m \, s^{-2}} $.

\subsection{Exercise 3b}
A unit of temperature $ T $ can be defined as $ T = \frac{\varepsilon}{k\sub{B}} $ with Boltzmann constant $ k\sub{B} $ in $ \unit{J \, K^{-1}} $.
A unit of pressure $ p $ can be defined as $ p = \frac{\varepsilon}{\sigma^3} $.

\subsection{Exercise 3c}
The unit of velocity can be defined as $ \frac{\sigma}{\tau} = \sqrt{\frac{\varepsilon}{m}} $.

\section{Exercises 4}
The centre of mass velocity $ v\sub{CM} $ is given by
\begin{equation}
	v\sub{CM} = \frac{\sum_{i = 1}^{N} m_i v_i}{ \sum_{i = 1}^{N} m_i },
\end{equation}
where $ v_i $ is the velocity of particle $ i $ with mass $ m_i $. 
In order to force the system to be stationary, for each direction, $ v\sub{CM} $ is calculated for the initial system with random velocities and is subtracted from each of the individual velocities so that a new calculation of $ v\sub{CM} $ yields $ v\sub{CM} = 0 $.
This is implemented in the code in the two for-loops starting at line 228.

\section{Exercise 5}
\subsection{Exercise 5a}
The force $ \vec{f} $ is calculated by
\begin{equation}
	\vec{f} = - \nabla U = 23 \varepsilon \left[ 2 \frac{\sigma^{12}}{r^{13}} - \frac{\sigma^6}{r^7} \right].
\end{equation}

\subsection{Exercise 5b}
For an $ N $-particle system, each particle interacts with $ N-1 $ other particles.
Therefore there are $ \frac{1}{2} N (N-1) $ interactions that have to be calculated.

\section{Exercise 6}
The Verlet algorithm is implemented in the code in the for-loop starting at line 348.

\section{Exercise 7}
\subsection{Exercise 7a}
This can be implemented in the code by subtracting or adding $ L $ to each of the coordinates of a particle independently such that $ 0 < x < L $, $ 0 < y < L $ and $ 0 < z < L $.
This could be done by using a modulus function or by using a floor function.

\subsection{Exercise 7b}
From $ 3^N -1 $, one finds that in $ 3 $-dimensions $ 26 $ mirror images are needed.

\subsection{Exercise 7c}
For interactions, the periodic boundary conditions can be implemented by subtracting or adding $ L $ such that $ -L/2 < \delta x < L/2 $, $ -L/2 < \delta y < L/2 $ and $ -L/2 < \delta z < L/2 $ where $ \delta x $, $ delta y $ and $ \delta z $ are the $ x $, $ y $ and $ z $ components of the distance between two interacting particles. 
This could be done by using a modulus function or by using a floor function.

\section{Problem 1}
The trajectories and the energy plots of a system with $ N = 2 $ particles is shown in \cref{fig:MDSLP1}.
\begin{figure}[h!]
	\centering
	\pgffigure{MDSLP1}
	\caption{Trajectories (left) and plot of the kinetic, potential and total energy (right) for a system with $ N = 2 $ particles.}
	\label{fig:MDSLP1}
\end{figure}

\section{Problem 2}
The trajectories and energy plots of a system with $ N = 3 $ particles is shown in \cref{fig:MDSLP2}.
\begin{figure}[h!]
	\centering
	\pgffigure{MDSLP2}
	\caption{Trajectories (left) and plot of the kinetic, potential and total energy (right) for a system with $ N = 3 $ particles.}
	\label{fig:MDSLP2}
\end{figure}

\section{Problem 3}
The trajectories and energy plots of systems with $ N = 4 $, $ N = 5 $ and $ N = 6 $ are shown in \cref{fig:MDSLP3N4,fig:MDSLP3N5,fig:MDSLP3N6}, respectively.

\begin{figure}[h!]
	\centering
	\pgffigure{MDSLP3N4}
	\caption{Trajectories (left) and plot of the kinetic, potential and total energy (right) for a system with $ N = 4 $ particles.}
	\label{fig:MDSLP3N4}
\end{figure}

\begin{figure}[h!]
	\centering
	\pgffigure{MDSLP3N5}
	\caption{Trajectories (left) and plot of the kinetic, potential and total energy (right) for a system with $ N = 5 $ particles.}
	\label{fig:MDSLP3N5}
\end{figure}

\begin{figure}[h!]
	\centering
	\pgffigure{MDSLP3N6}
	\caption{Trajectories (left) and plot of the kinetic, potential and total energy (right) for a system with $ N = 6 $ particles.}
	\label{fig:MDSLP3N6}
\end{figure}

\section{Problem 4}
In order to implement the velocity Verlet scheme, the function \lstinline[basicstyle=\ttfamily]|time_step_verlet| is replaced. 
The new version of this function is shown in \cref{lst:velocityVerlet}. 
In the implementation, first for every particle the new position and the mid-step velocity are calculated.
After these calculations, the acceleration for every particle is calculated (this is done by the function \lstinline[basicstyle=\ttfamily]|accelerations|).
With these new accelerations, the final velocity is calculated.

\begin{lstlisting}[language = C, basicstyle=\ttfamily\small,  caption = {Implementation of the velocity Verlet scheme}, label = {lst:velocityVerlet}]
void time_step_verlet( int iprint )
{
	int i, j;

	for ( i = 0 ; i<Npart ; i++ )
	{
		// Calculate the new position for every particle
		R[i].x += dt*V[i].vx + dt*dt*acc[i].ax*0.5;
		R[i].y += dt*V[i].vy + dt*dt*acc[i].ay*0.5;
		R[i].z += dt*V[i].vz + dt*dt*acc[i].az*0.5;
		
		// Calculate the mid-step velocity for every particle
		V[i].vx += dt*acc[i].ax*0.5;
		V[i].vy += dt*acc[i].ay*0.5;
		V[i].vz += dt*acc[i].az*0.5;
	}
	
	// Calculate accelerations at timestep "t+dt"
	accelerations( iprint );  
	
	for ( i = 0 ; i<Npart ; i++ )
	{
		V[i].vx += dt*acc[i].ax*0.5;
		V[i].vy += dt*acc[i].ay*0.5;
		V[i].vz += dt*acc[i].az*0.5;
	}
}
\end{lstlisting}

\section{Problem 5}
The trajectories and energies of the systems with $ N = 2 $ and $ N = 3 $ particles, simulated using the velocity Verlet scheme, are shown in \cref{fig:MDSLP5N2,fig:MDSLP5N3}.
The trajectories for the $ N = 3 $ particle system differ slightly from the earlier simulation, which used the ordinary Verlet scheme, which is shown in \cref{fig:MDSLP2}.
Moreover, the total energy of the $ N = 3 $ particles system as calculated using the velocity Verlet scheme shows less fluctuations compared to the total energy as calculated using the ordinary Verlet scheme, which indicates that the velocity Verlet scheme is more accurate than the regular Verlet scheme.

\begin{figure}[h!]
	\centering
	\pgffigure{MDSLP5N2}
	\caption{Trajectories (left) and plot of the kinetic, potential and total energy (right) for a system with $ N = 2 $ particles calculated using the velocity Verlet scheme.}
	\label{fig:MDSLP5N2}
\end{figure}

\begin{figure}[h!]
	\centering
	\pgffigure{MDSLP5N3}
	\caption{Trajectories (left) and plot of the kinetic, potential and total energy (right) for a system with $ N = 3 $ particles calculated using the velocity Verlet scheme.}
	\label{fig:MDSLP5N3}
\end{figure}

\section{Problem 6}
\todo{Problem 6 maken}

\section{Problem 7}
\todo{Problem 7 maken}

\section{Problem 8}
\begin{figure}[h!]
	\centering
	\pgffigure{MDSLP8N12}
	\caption{\todo{add caption}}
	\label{fig:MDSLP8N12}
\end{figure}

\begin{figure}[h!]
	\centering
	\pgffigure{MDSLP8N24}
	\caption{\todo{add caption}}
	\label{fig:MDSLP8N24}
\end{figure}

\todo{Problem 8 maken}

\section{Problem 9}
\todo{Problem 9 maken}

\section{Master exercise 1}
\todo[inline]{choose 1 of the master exercises}
\todo[inline]{exercise met kracht uitrekenen vector van maken en getal narekenen}


\chapter{Gromacs}
\section{Argon}
\begin{figure}[h!]
	\centering
	\pgffigure{ArgonColdDensTemp}
	\caption{\todo{add caption}}
	\label{fig:ArgonColdAnnealedDensTemp}
\end{figure}

\begin{figure}[h!]
	\centering
	\pgffigure{ArgonAnnealedDensTemp}
	\caption{\todo{add caption}}
	\label{fig:ArgonAnnealedDensTemp}
\end{figure}

\begin{figure}[h!]
	\centering
	\pgffigure{ArgonMSD}
	\caption{\todo{add caption}}
	\label{fig:ArgonMSD}
\end{figure}

\begin{figure}[h!]
	\centering
	\pgffigure{ArgonRDF}
	\caption{\todo{add caption}}
	\label{fig:ArgonRDF}
\end{figure}

\section{Chloroform}

\begin{figure}[h!]
	\centering
	\pgffigure{CFMColdDensTemp}
	\caption{\todo{add caption}}
	\label{fig:CFMColdAnnealedDensTemp}
\end{figure}

\begin{figure}[h!]
	\centering
	\pgffigure{CFMAnnealedDensTemp}
	\caption{\todo{add caption}}
	\label{fig:CFMAnnealedDensTemp}
\end{figure}

\begin{figure}[h!]
	\centering
	\pgffigure{CFMMSD}
	\caption{\todo{add caption}}
	\label{fig:CFMMSD}
\end{figure}

\begin{figure}[h!]
	\centering
	\pgffigure{CFMRDF}
	\caption{\todo{add caption}}
	\label{fig:CFMRDF}
\end{figure}



\insertbibliography
%\begin{appendices}
%% appendices hier
%
%
%\end{appendices}


\end{document}